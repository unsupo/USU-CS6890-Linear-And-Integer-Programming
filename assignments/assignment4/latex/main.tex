\documentclass[11pt,a4paper,openany]{report}

\title{CS6890 HW04}
\author{Jonathan Arndt }
\date{July 31 2017}

\usepackage{amsmath,mathtools}
\usepackage{systeme}
\usepackage[utf8]{inputenc}
\usepackage{newunicodechar}
\usepackage{amsmath}
\usepackage[english]{babel}
\usepackage{listings}


%% the first is the “unknown minus” (U+2212), the second is a hyphen
\newunicodechar{−}{-}

\newenvironment{sysmatrix}[1]
 {\left(\begin{array}{@{}#1@{}}}
 {\end{array}\right)}
\newcommand{\ro}[1]{%
  \xrightarrow{\mathmakebox[\rowidth]{#1}}%
}
\newlength{\rowidth}% row operation width
\AtBeginDocument{\setlength{\rowidth}{3em}}

\begin{document}

\maketitle

\tableofcontents
\newpage
\section{Problems}
\subsection{Problem 1}
See TableauSimplex.java attached.

\subsection{Problem 2}
Utilizes TableauSimplex.java
\begin{verbatim}
2000acre
A and B crops
A: 1 person-day of labor and $90 of capital for each acre planted
B: 2 person-day of labor and $60 of capital for each acre planted

A produces $170 revenue/acre
B produces $190 revenue/acre

maximize 170A+190B
A:  1+2   <= 3000
B:  90+60 <= 150000

System.out.println(
  getMatrixString(
      TableauSimplex.solveSimplexTableau(
              new double[][]{
                  {1,2,1,0,3000},
                  {90,60,0,1,150000},
                  {-170,-190,0,0,0}
              }
            ),new String[][]{
                  {"x1", "x2", "s1", "s2", "P"},
                  {"Crop A", "Crop B","Profit"}
        })
  );

Produces this output:

        x1  x2  s1     s2      |  P
Crop A  0   1   3/4    -1/120  |  1000
Crop B  1   0   -1/2   1/60    |  1000
-------------------------------------------
Profit  0   0   115/2  5/4     |  360000

\end{verbatim}
\subsubsection{Part 1.}
\begin{verbatim}
100 more person-days:

System.out.println(
  getMatrixString(
      TableauSimplex.solveSimplexTableau(
              new double[][]{
                  {1,2,1,0,3100},
                  {90,60,0,1,150000},
                  {-170,-190,0,0,0}
              }
            ),new String[][]{
                  {"x1", "x2", "s1", "s2", "P"},
                  {"Crop A", "Crop B","Profit"}
        })
  );

Produces this output:

        x1  x2  s1     s2      |  P
Crop A  0   1   3/4    -1/120  |  1075
Crop B  1   0   -1/2   1/60    |  950
-------------------------------------------
Profit  0   0   115/2  5/4     |  365750
\end{verbatim}

\subsubsection{Part 2.}
\begin{verbatim}
$100 more available capital

System.out.println(
  getMatrixString(
      TableauSimplex.solveSimplexTableau(
              new double[][]{
                  {1,2,1,0,3000},
                  {90,60,0,1,150100},
                  {-170,-190,0,0,0}
              }
            ),new String[][]{
                  {"x1", "x2", "s1", "s2", "P"},
                  {"Crop A", "Crop B","Profit"}
        })
  );

Produces this output:

        x1  x2  s1     s2      |  P
Crop A  0   1   3/4    -1/120  |  6445/6
Crop B  1   0   -1/2   1/60    |  2855/3
-------------------------------------------
Profit  0   0   115/2  5/4     |  365875
\end{verbatim}

\subsubsection{Part 3.}
Baseline:                     360000\\
Increase person-hours by 100: 365750\\
Increase capital by \$100:    365875\\



\subsection{Problem 3}
See TableauSimplex.java
\begin{verbatim}
Large and Small muffins
Large: 4oz dough, 2oz bran
Small: 1oz dough, 1oz bran
300oz dough, 160oz bran

maximize .25L+.1S
4+1<=300
2+1<=160

System.out.println(getMatrixString(
    TableauSimplex.solveSimplexTableau(
            new double[][]{
                {4,1,1,0,300},
                {2,1,0,1,160},
                {-.25,-.1,0,0,0}
            }
    ),new String[][]{
            {"x1", "x2", "s1", "s2", "P"},
            {"Large", "Small","Profit"}
    }));

Results in:
        x1  x2  s1    s2    |  P
Large   1   0   1/2   -1/2  |  70
Small   0   1   -1    2     |  20
--------------------------------------
Profit  0   0   1/40  3/40  |  39/2
\end{verbatim}
\subsubsection{Part 1.}
\begin{verbatim}
double d = Double.MIN_VALUE;
int bran = 300;
while(true){
    Ratio[][] tableau = TableauSimplex.solveSimplexTableau(
            new double[][]{
                    {4, 1, 1, 0, bran++},
                    {2, 1, 0, 1, 160},
                    {-.25, -.1, 0, 0, 0}
            }
    );
    double q;
    if((q=tableau[tableau.length-1][tableau[0].length-1].getDoubleValue()) <= d)
        break;
    d = q;
}
System.out.println("Bran max with dough 160 is: "+(bran-1));

Results in:
Bran max with dough 160 is: 321
\end{verbatim}
\subsubsection{Part 2.}
\begin{verbatim}
double d = Double.MIN_VALUE;
int dough = 160;
while(true){
    Ratio[][] tableau = TableauSimplex.solveSimplexTableau(
            new double[][]{
                    {4, 1, 1, 0, 300},
                    {2, 1, 0, 1, dough++},
                    {-.25, -.1, 0, 0, 0}
            }
    );
    double q;
    if((q=tableau[tableau.length-1][tableau[0].length-1].getDoubleValue()) <= d)
        break;
    d = q;
}
System.out.println("Bran max with dough 160 is: "+(dough-1));

Results in:
Bran max with dough 160 is: 301

\end{verbatim}

\subsection{Problem 4}
See TableauSimplex.java
\begin{verbatim}
250mg of Calcium
500mg of phosphorous
9mg of iron
apples,orances,bannanas

minimize calories

minimize 60a+50o+90b
calcium:      10a+40o+60b = 250
phosphorous:  10a+20o+30b = 500
iron:         .3a+.2o+.6b = 9

10a+40o+60b <= 250
-10a-40o-60b <= 250
10a+20o+30b <= 500
-10a-20o-30b <= 500
.3a+.2o+.6b <= 9
-.3a-.2o-.6b <= 9


System.out.println(getMatrixString(
    TableauSimplex.solveSimplexTableau(
            new double[][]{
                    {10,40,60,1,0,0,0,0,0,250},
                    {-10,-40,-60,0,1,0,0,0,0,-250},
                    {10,20,30,0,0,1,0,0,0,500},
                    {-10,-20,-30,0,0,0,1,0,0,-500},
                    {.3,.2,.6,0,0,0,0,1,0,9},
                    {-.3,-.2,-.6,0,0,0,0,0,1,-9},
                    {-60,-50,-90,0,0,0,0,0,0,0}
            }
    ),new String[][]{
            {"Apples", "Oranges", "Bananas", "s1","s2","s3","s4","s5","s6", "P"},
            {"Calcium", "-Calcium","Phosphorus","-Phosphorus","Iron","-Iron","Calories"}
    }));

Results in:
             Apples  Oranges  Bananas  s1      s2  s3  s4  s5  s6  |  P
Calcium      1       4        6        1/10    0   0   0   0   0   |  25
-Calcium     0       0        0        1       1   0   0   0   0   |  0
Phosphorus   0       -20      -30      -1      0   1   0   0   0   |  250
-Phosphorus  0       20       30       1       0   0   1   0   0   |  -250
Iron         0       -1       -6/5     -3/100  0   0   0   1   0   |  3/2
-Iron        0       1        6/5      3/100   0   0   0   0   1   |  -3/2
-----------------------------------------------------------------------------
Calories     0       190      270      6       0   0   0   0   0   |  1500


No Apples, 190 oranges and 270 bananas

\end{verbatim}

\newpage
\section{Source Code}
\subsection{TableauSimplex.java}
\lstinputlisting[language=Java]{../../../Linear-And-Integer-Programming/src/main/java/assignment/TableauSimplex.java}
\newpage
\subsection{Ratio.java}
\lstinputlisting[language=Java]{../../../Linear-And-Integer-Programming/src/main/java/utilities/Ratio.java}


\end{document}
